\chapter{Introduction}
\label{Intro}

\section{Introduction}
This report is made in the course ITAMS on Engineering Collage of Aarhus. It is made by Nicolai Glud and Johnny Kristensen and describes the project outlined by the group.

\section{What is Geocaching?}
Geocaching is a modern form of treasure hunt. The goal of geocaching is to find hidden "caches" which other people have hidden. Data and puzzles about the caches can be found on a website og via. various apps to phones, tablets and pc's. Normally some GPS coordinates are given and from that you have to find the "cache".\\
The following cite is from www.geocaching.com:\\
"Geocaching is a real-world, outdoor treasure hunting game using GPS-enabled devices. Participants navigate to a specific set of GPS coordinates and then attempt to find the geocache (container) hidden at that location."

\begin{wrapfigure}{r}{5cm}
\vspace{-30pt}
\begin{center}
\includegraphics[width=4cm]{billeder/geodude}
\end{center}
\vspace{-20pt}
\caption{Geodude handheld device}
\vspace{-20pt}
\end{wrapfigure}

\section{Full featured system}

The full featured system is meant to be a fully blown geocaching dedicated device. We imagine features like being able to plot the route you were just on, via your PC at home or you could upload several cache locations into the system so it automatically points you to the nearest cache.\\
The system will have several different screens or modes that you can cycle through. One which displays the location and distance to the nearest cache, one that displays just your heading (like a compass), one that displays your current location and heading and lastly one that shows your route. More screens can be added. \\
A keypad lets you change settings while using the system out in the field. This could be to change the route, timezone or even which cache you are going for.\\

\section{Project boundaries}
%% Her har vi næsten gennemgået nogle af de technical considerations som han gerne vil have ?? %%%%%%%%%%%%%
The full featured system would take long time to implement and we only have 3 weeks to built as many features as possible. We have chosen a series of features that will go into our basic implementation of the system. These features are:
Fully working screen with 1 mode: "Current location, time and heading".\\
Fully working heading driver.\\
GPS driver for the screen mode selected. \\
To handle the data and communication we have chosen the Atmel ATMega32 microcontroller.\\

\section{Referenced documentation}
The datasheets for the modules can be found in appendix.

\section{Glossary \& abbreviations}







