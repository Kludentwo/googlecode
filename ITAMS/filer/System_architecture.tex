\chapter{System architecture}
The system architecture is explained in the following chapter.

\section{System description}
The system consists of the following 4 subsystem: The magnetometer, the screen, the GPS module and the ATmega32. Together they form the basic GeoDude system. With the basic system in place, it can show you your current gps location, the time and your heading.\\


\subsection{System block diagram}
The General system can be illustrated by the following block definition diagram:\\
\begin{figure}[H]
	\centering
	\includegraphics[width=.5\textwidth]{billeder/SystemBDD}
	\caption{Block Definition Diagram of the system}
	\label{bdd:system}
\end{figure}

\section{Behavioural description}
Below is shown a statemachine describing the systems functionality.
\begin{figure}
\centering
\includegraphics[width=.8\textwidth]{billeder/geodudeSTM}
\caption{GeoDude statemachine}
\end{figure}
The scheduler is a part of FreeRTOS. The ticks inside the scheduler determines how often a task is called. When the boot loading task is called and the button is activated the jump down to the boot loading section and FreeRTOS stops. After boot loading, the system is reset and the state machine starts from the beginning.

\section{Interfaces/pin assignments}
GeoDude has 3 interfaces, SPI to the screen, I$^2$C to the magnetometer and UART to the GPS receiver.
The following pin assignments have been used:
\begin{table}[H]
\centering
    \begin{tabular}{|l|l|l|l|}
    \hline
    PIN 		& Description    & Levels 	& Unit  \\ \hline
    VCC (10) 	& Supply         & 5 V    	& -		\\ \hline
    GND (11) 	& Ground         & Ground  	& -		\\ \hline
    PA0 (40)	& D/C			 & 0 - 3 V	& Screen \\ \hline
    PA1 (39)	& CS			 & 0 - 3 V  & Screen \\ \hline
    PA2 (38)	& RESET(RST)	 & 0 - 3 V  & Screen \\ \hline
    PB5	(06)	& SPI MOSI		 & 0 - 3 V  & Screen \\ \hline
    PB7 (08)	& SPI CLK		 & 0 - 3 V  & Screen \\ \hline
    PC0 (22) 	& I2C Clock line & 0 - 5 V 	& Magnetometer 	\\ \hline
    PC1 (23) 	& I2C Data line  & 0 - 5 V 	& Magnetometer		\\ \hline
    PD0 (14)	& UART RXD		 & 0 - 5 V	& GPS module and Bootloading \\ \hline
    PD1 (15)    & UART TXD		 & 0 - 5 V  & Bootloading \\ \hline
    \end{tabular}
    \caption{Geodude pin assignments}
\end{table}
When boot loading, the GPS module is powered down and disconnected.


%\subsection{ATmega32 block diagram}
%The General system can be by the following block definition diagram:\\
%\begin{figure}[H]
%	\centering
%	\includegraphics[width=.8\textwidth]{billeder/atmegaBDD}
%	\caption{Block Definition Diagram of the ATmega32}
%	\label{bdd:system}
%\end{figure}
%
%This will lead us to use the following pins:\\
%Logic levels are written as 0 - 5 V.\\
%
%\begin{table}[H]
%\centering
%    \begin{tabular}{|l|l|l|l|}
%    \hline
%    PIN 		& Description    & Levels 	& Unit  \\ \hline
%    VCC (10) 	& Supply         & 5 V    	& -		\\ \hline
%    GND (11) 	& Ground         & Ground  	& -		\\ \hline
%    PA0 (40)	& D/C			 & 0 - 5 V	& Screen \\ \hline
%    PA1 (39)	& CS			 & 0 - 5 V  & Screen \\ \hline
%    PA2 (38)	& RESET(RST)	 & 0 - 5 V  & Screen \\ \hline
%    PB5	(06)	& SPI MOSI		 & 0 - 5 V  & Screen \\ \hline
%    PB7 (08)	& SPI CLK		 & 0 - 5 V  & Screen \\ \hline
%    PC0 (22) 	& I2C Clock line & 0 - 5 V 	& Magnetometer 	\\ \hline
%    PC1 (23) 	& I2C Data line  & 0 - 5 V 	& Magnetometer		\\ \hline
%    PD0 (14)	& UART RXD		 & 0 - 5 V	& GPS module \\ \hline
%    \end{tabular}
%\end{table}
%
%\subsection{Peripherals}
%Logic levels are written as 0 - 5 V or 0 - 3 V.\\
%
%\textbf{The magnetometer:}\\
%\begin{table}[H]
%\centering
%    \begin{tabular}{|l|l|l|}
%    \hline
%    PIN & Description    & Levels  \\ \hline
%    SCL & I2C Clock line & 0 - 5 V \\ \hline
%    SDA & I2C Data line  & 0 - 5 V \\ \hline
%    VCC & Supply         & 5 V     \\ \hline
%    GND & Ground         & Ground  \\ \hline
%    \end{tabular}
%\end{table}
%
%\textbf{The screen:}\\
%\begin{table}[H]
%\centering
%    \begin{tabular}{|l|l|l|}
%    \hline
%    PIN & Description    & Levels  \\ \hline
%    CLK & SPI Clock line & 0 - 3 V \\ \hline
%    DIN & Data in (or MOSI)  & 0 - 3 V \\ \hline
%    VCC & Supply         & 3 V     \\ \hline
%    GND & Ground         & Ground  \\ \hline
%    D/C & Data or Command & 0 - 3 V \\ \hline
%    CS  & Chip Select    & 0 - 3 V \\ \hline
%    RST & Reset			 & 0 - 3 V \\ \hline
%    LED & Background light on display & 3 V \\ \hline
%    \end{tabular}
%\end{table}
%The screen runs on 3 V VCC so all communication from the ATmega32 will have to be voltage divided. 
%
%\textbf{The gps:}\\
%\begin{table}[H]
%\centering
%    \begin{tabular}{|l|l|l|}
%    \hline
%    PIN & Description    & Levels  \\ \hline
%    TX & UART transfer data  & 0 - 5 V \\ \hline
%    VCC & Supply         & 5 V     \\ \hline
%    GND & Ground         & Ground  \\ \hline
%    \end{tabular}
%\end{table}