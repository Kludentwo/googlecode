\chapter{Introduction}


\section{Introduction}
This report is made in the curse ITAMS on Engineering Collage of Aarhus. It is made by Nicolai Glud and Johnny Kristensen and describes the project outlined by the group.

\section{What is Geocaching?}
Geocaching is a modern form of treasure hunt. The goal of geocaching is to find hidden "caches" which other people have hidden. Data and puzzles about the caches can be found on a website og via. various apps to phones, tablets and pc's. Normally some GPS coordinates are given and from that you have to find the "cache".\\
The following cite is from www.geocaching.com:\\
"Geocaching is a real-world, outdoor treasure hunting game using GPS-enabled devices. Participants navigate to a specific set of GPS coordinates and then attempt to find the geocache (container) hidden at that location."

\begin{wrapfigure}{r}{5cm}
\vspace{-30pt}
\begin{center}
\includegraphics[width=4cm]{billeder/geodude}
\end{center}
\vspace{-20pt}
\caption{Geodude handheld device}
\vspace{-20pt}
\end{wrapfigure}

\section{Full featured system}

The full featured system is meant to be a fully blown geocaching dedicated device. We imagine features like being able to plot the route you were just on, via your PC at home, to code several caches in so the system automatically can point to the nearest cache. 



\section{Project description}

\section{Referenced documentation}

\section{Glossary \& abbreviations}

\section{Introduction}
GeoDude is a project with geocaching in mind. The project is developed in respect to the course ITAMS. Geocaching is a type of treasure hunt where you look for caches with a set of coordinates. By making a handheld unit that can get coordinates and your heading we seek to create a device useful for modern day treasure hunters. 

\section{Project description}
%%Beskrivelse af hvordan projektet er tænkt, og i hvilket omfang det er tænkt brugt.
Er det ikke det der står i Introduction? Skal det flyttes eller hvad?

\section{Project boundaries}
Although the 

\section{Datasheets}
The following datasheets have been used for the following components:\\
Magnetometer: HMC6352.pdf\\
Screen: Nokia5110.pdf\\
GPS module: RGM-2000\_user\_manual.pdf\\

The datasheets can be found in appendix.

\section{Glossary \& abbreviations}
\begin{table}[H]
\centering
\begin{tabular}{|p{4cm}|p{7cm}|}
\hline
Term/abbreviation & Definition\\ \hline
\end{tabular}
\end{table}

