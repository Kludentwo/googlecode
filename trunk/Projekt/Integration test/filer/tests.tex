\chapter{Integration Tests}

\section{Test case scenarios}
\begin{table}[H]
    \begin{tabular}{|l|l|l|p{10cm}|}
    \hline
    Case & Control Unit & Unit under Test & Execution \\ \hline
    1 & CDU & Sensor & CDU sends a get data request to a sensor. \\ \hline
    2 & Sensor & CDU & Sensor sends data to the CDU \\ \hline
    3 & PC & CDU & PC sends a get data request to the CDU \\ \hline
    4 & CDU & Sensor & CDU requests data from a sensor that does not exists \\ \hline
    5 & PC & CDU & PC sends a message to the CDU that the CDU does not recognise. \\ \hline
    6 & Sensor & CDU & Sensor sends invalid data to the CDU \\ \hline
    7 & Sensor & CDU & CDU sends a get data request to a sensor but does not allow the sensor to respond. \\ \hline
    8 & Sensor & CDU & CDU sends a get data request to a sensor. The sensor responds but occupies the bus and never lets go. \\ \hline
    \end{tabular}
\end{table}

\section{Test case results}
\begin{table}[H]
    \begin{tabular}{|l|l|l|p{10cm}|}
    \hline
    Case & Expected result & Actual result & Done \\ \hline
    1 & ~ & ~ & ~\\ \hline
    2 & ~ & ~ & ~\\ \hline
    3 & ~ & ~ & ~\\ \hline
    4 & ~ & ~ & ~\\ \hline
    5 & ~ & ~ & ~\\ \hline
    6 & ~ & ~ & ~\\ \hline
    7 & ~ & ~ & ~\\ \hline
    8 & ~ & ~ & ~\\ \hline
    \end{tabular}
\end{table}

\chapter{Unit Tests}
\section{CDU}
%% The hardware tests of the CDU
This section contains the hardware and software unit tests of the CDU.
\subsection{Hardware}
In order to test the hardware of the CDU, a series of test points was put on the circuitry. External hardware will be needed for some tests. This will be stated clearly if needed.
\subsubsection{Test case 1: 3.3 volt power supply}
\textbf{Purpose:}\\
The purpose of this test is to test the 3.3 volt circuitry of the power supply block in the CDU.\\

\textbf{Procedure:}\\
Apply 20 V on the P+ and P- pins. Measure voltage between 3v3TEST pin and GNDPIN with an oscilloscope.\\

\textbf{Expected Result:}\\
3.35$\pm$0.17 volts are measured.\\

\textbf{Actual Result:}\\
3.36 volts\\

\textbf{Comment and remarks:}\\
Well in range\\

\subsubsection{Test case 2: Communication to bus}
\textbf{Purpose:}\\
The purpose of this test is to test the communication to bus circuitry.\\

\textbf{Procedure:}\\
Apply 20 V on the P+ and P- pins. Between B+ and B- pins put a circuit of 4.7 volt Zener diode and a 5 ohm resistor. Apply 10 kHz signal to INPUTTEST pin. Observe AC coupled signal between SenseTEST pin and GNDPIN with an oscilloscope.\\

\textbf{Expected Result:}\\
10 kHz signal is observed.\\

\textbf{Actual Result:}\\
10 kHz signal is observed.\\

\textbf{Comment and remarks:}\\
-\\

\subsubsection{Test case 3: Voltage reference in receiver circuit}
\textbf{Purpose:}\\
The purpose of this test is to test the voltage reference in receiver circuitry.\\

\textbf{Procedure:}\\
Apply 20 V on the P+ and P- pins. Between B+ and B- pins put a circuit of 4.7 volt Zener diode and a 5 ohm resistor. Measure voltage between REFTEST pin and GNDPIN with an oscilloscope.\\

\textbf{Expected Result:}\\
0.43$\pm$0.02 volts are measured.\\

\textbf{Actual Result:}\\
0.44 volts are measured.\\

\textbf{Comment and remarks:}\\
within range\\

\subsubsection{Test case 4: Communication from bus}
\textbf{Purpose:}\\
The purpose of this test is to test the communication from bus circuitry\\

\textbf{Procedure:}\\
Apply 20 V on the P+ and P- pins. Between B+ and B- pins put a circuit of 4.7 volt Zener diode, a 5 ohm resistor and a BAV21 diode and ZVN3306A. For clarification look at figure ~\ref{fig:busteststub}. Apply 10 kHz signal to the ZVN3306A on the pins shown in the bus test stub figure. Observe AC signal between OUTPUTTEST pin and GNDPIN with an oscilloscope.\\
\begin{figure}[H]
\centering
\includegraphics[width=0.8\textwidth]{billeder/BusTestStub}
\caption{Bus test stub}
\label{fig:busteststub}
\end{figure} 


\textbf{Expected Result:}\\
10 kHz signal is observed.\\

\textbf{Actual Result:}\\
\\

\textbf{Comment and remarks:}\\
-\\

\subsection{Software}
The software tests of the CDU

\section{Sensor Node}
\subsection{Hardware}
The hardware tests of the Sensor node
\subsubsection{Test case 5: 3.3 volt power supply}
\textbf{Purpose:}\\
The purpose of this test is to test the 3.3 volt circuitry of the power supply block in the sensor node.\\

\textbf{Test equipment:}
\begin{itemize}
\item Multimeter: Meterman 37XR
\item Sensor node (UUT)
\item Sensor node supply test stub
\end{itemize}
\ \\
\textbf{Procedure:}\\
Connect B+ from the sensor node supply test stub (see figure \ref{fig:sensor_node_supply_test_Stub}) to the S+ on the UUT. Connect B- from the sensor node supply test stub to the S- on the UUT.\\
Measure the voltage from the GND testpin to the 3V3 testpin.

\begin{figure}[H]
	\centering
	\subbottom[Sensor node supply test stub]{%
		\includegraphics[width=.48\linewidth]{billeder/sensor_node_supply_test_stub}}
	\subbottom[Picture of the sensor node]{%
		\includegraphics[width=.48\linewidth]{billeder/case5_picture}}

%\label{fig:sensor_node_supply_test_Stub}
%\end{subfigure}
\end{figure}

\textbf{Expected Result:}\\
3.35V$\pm$0.17V are measured.\\

\textbf{Result:}
\begin{table}[H]
\centering
\begin{tabular}{|p{2cm}|p{2cm}|p{3cm}|p{2cm}|}\hline
\textbf{Test case:} & \textbf{Date:} & \textbf{Measurement:} & \textbf{Result:} \\ \hline
5 & 30-04-2014 & 3.30V & PASSED \\ \hline
\end{tabular}
\end{table}

\textbf{Comment and remarks:}\\
-\\

\subsubsection{Test case 6: Data recovery circuit}
\textbf{Purpose:}\\
The purpose of this test is to test the data recovery circuit located on the sensor node. The levels and rise time must not vary more then specified in the implementation documentation.\\

\textbf{Test equipment:}
\begin{itemize}
\item HAMEG HM203-6 Oscilloscope
\item HAMEG HM8030-3 Function generator
\item Sensor node (UUT)
\item Sensor node communication stub
\end{itemize}
\ \\
\textbf{Procedure:}\\
Connect B+ from the sensor node communication test stub to the S+ on the UUT. Connect B- from the sensor node communication test stub to the S- on the UUT.
Connect FG+ from the sensor node communication test stub to the positive connector on the Function generator. Connect FG- from the sensor node communication test stub to the negative connector on the Function generator.\\ Set the Function generator to:
\begin{itemize}
\item pk-pk: 3.3V$\pm$0.17V
\item DC offset: pk-pk/2
\item frequency: 10kHz
\item Waveform: Square
\end{itemize}
Connect the oscilloscope ground to the GND test point on the UUT.
Connect the oscilloscope probe to the DIN point on the UUT.
Set the Oscilloscope to:
\begin{itemize}
\item Volts/div: 1V
\item Time/div: 50$\mu$s
\item Probe: AC coupled
\end{itemize}
\begin{figure}[H]
	\centering
	\subbottom[Sensor node supply test stub]{%
		\includegraphics[width=.48\linewidth]{billeder/sensor_node_communication_test_stub}}
	\subbottom[Picture of the sensor node]{%
		\includegraphics[width=.48\linewidth]{billeder/case6_picture}}

%\label{fig:sensor_node_supply_test_Stub}
%\end{subfigure}
\end{figure}

\textbf{Expected Result:}\\
Frequency: 10kHz$pm$0.5kHz\\
Rise time: <10$\mu$s\\
Store a picture of the oscilloscope


\textbf{Result:}
\begin{table}[H]
\centering
\begin{tabular}{|p{2cm}|p{2cm}|p{3cm}|p{2cm}|}\hline
\textbf{Test case:} & \textbf{Date:} & \textbf{Measurement:} & \textbf{Result:} \\ \hline
5 & 30-04-2014 & 3.30V & PASSED \\ \hline
\end{tabular}
\end{table}
\textbf{Picture:}
\textbf{Comment and remarks:}\\
-\\
\subsection{Software}
The software tests of the Sensor node