\chapter{Further development}
This chapter seeks to explain the further development needed for this project to become a full product.\\

The whole project was made with discrete components that were available from the schools electronics laboratory. This means that the prototype was developed with components that were not the most fitting for the project. To further improve the project some components will have to be changed for cheaper or more effective components or circuits.\\

A full battery supply system will need to be designed. It must take the total power consumption over time in consideration in order to comply with the requirement specified in the original project formulation. In order to comply with the low power demands of battery using applications, the device will need components working at a lower voltage. The smaller the voltage needed by the bus, the smaller the battery needed to power the whole system. The battery must also be rechargeable which further limits the power consumption by the total system.\\

A big part of the further development is the design and synthesis of an integrated circuit. The whole sensor node must be redesigned with the same blocks but different components and layouts. The finished sensor node IC will still need to have a small circuit board with a few capacitors to ensure stability.\\

Another part is the application layer of the CDU. The functions and descriptions have been made but the application will need to be written and optimised for this project to become an actual product. Further development on the CDU could be the ability to update the firmware through a connected placed somewhere in the project.\\

The measurement system will also have to be improved. Either through a redesign or implementation of a know system with low power consumption and low noise. This could be a full project on its own.

\textbf{Improvements of design areas}.\\
