\chapter{Project Formulation}
\textit{This chapter describes the project formulation forming the foundation for the project. The original project formulation was provided by DELTA and adopted by the group.}
\section{Original project formulation}
This is only a short excerpt of the original project formulation, for the entire original project description see the enclosed CD \cite{cd}.
\subsection{Background}
A small part of diabetic patients will develop Charcot-foot, a complication often introduced late in the illness. This complication immobilises the patient and reduces quality of life.\\
It is believed that the charcot-foot is causing a rise in the temperature of the afflicted foot and that the heat signature is different from a healthy foot. Earlier projects have shown that temperature measurement can be used to evaluate on the condition of the foot in the rehabilitation phase.

\subsection{Project description}
The project should end in a prototype able to continually measure the temperature of the foot. The device should be worn by the patient throughout the daily life.\\
The project is expected to deliver a design of such a system, where focus should be on the temperature measurement device, although it is expected to describe the entire system.\\
Expected results:
\begin{itemize}
	\item Prototype
	\item Requirement specification of prototype
	\item System description including interface descriptions
\end{itemize}


\section{Modified project formulation}
This project description is an extension of the original project description. 
\subsection{Background}
The original project formulation was not fulfilling to the group with regards to new knowledge and technology. Therefore the project was modified to accommodate the groups needs. The change of the project has been made in collaboration with the supervisor.\\
The extension includes new communication protocol development and smart-sensor development.\\
Therefore this project will lead more towards a technology development rather than a product development.\\

\subsection{Project description}
The project should end in a prototype comprised of discrete component on a PCB with intentions of synthesizing it into an ASIC(for the sensor nodes). The project should make a proof of concept on a power line communication bus reducing wires needed for the system.\\
The system will be comprised of a number of sensor nodes daisy chained to a central data unit. The central data unit will address the sensor nodes to extract the data and store it locally.\\
The modified system seeks to provide the technology for the desired system requested in original project description. The daisy chained sensors will limit the number of wires needed in the sock or shoe. The hypothesis that fewer wires cause less discomfort and makes it easier to build the end product is established.\\

