\chapter{Conclusion}
The development methods used in this project led to a clear progression through the phases. Multiple iterations meant the group had to evaluate the current state of the solution and think of improvements to the design. The time schedule was modified during the project to better reflect the most important phase at the time. There is no clear division of the project work as the group members have worked closely together.

In relation to the project scope the group has achieved a working central data collection unit referenced as the CDU. The CDU has a working PC interface. Multiple data acquisition nodes referenced as the sensor nodes have been made with working measurement system. The two components of the system can successfully communicate using the custom power line communication bus. The group has achieved two way communication. The sensor nodes are powered by the bus that consist of a single wire going through each sensor.

With regards to the requirements specification the group has achieved the ability to save data entries correctly in memory. The memory is not large enough to fit 24 hours of data collection but serves as a proof of concept. The requirement for power consumption will be a part of the further development phase along with the optimisation of the design.

The whole system has been build as a prototype on PCBs with discrete components. In addition to a PCB, the CDU uses an Explorer 16 development board. The sensor node utilises an Altera DE2 board.

The technology developed in this project is a communication system built on top of a current loop. It serves as a proof of concept that is is physically possible to create a bus with a single wire daisy chaining multiple sensor nodes. The bus is stable and reliable. 

The further development chapter explains the work needed for this project to become a working product using our technology and design. The group is proud to conclude on the study and the project as a whole.