\chapter{Obtained experience}

\section{Personal experience}
Working on this project has given the members of the group a good understanding for hardware components and how to utilise off the shelf components to create simple designs and solutions. The project has been far more in depth hardware development then any previous project, and has given us great experience with actual hardware development. This has also shown all the littlest things can intervene and disturb. Even though we though the current loop was very suppressing to EMC and noise, in the end we found that this was the source to a lot of our problems.\\
Thereby hardware debugging has been a big part of the development phase and has help to give a natural understanding of where to measure, redesign and rethink.

\section{Technology studies experience}
Previous project have had little or next to no technology studies involved. This phase has help give a better approach when searching for new technologies and utilizing them in a project. As it is also seen in the calendar schedule this phase takes a lot of time, this is part finding the technology needed and part understanding the technology and how to use it. 

\section{Development issues}
Throughout the project some hurdles and issues appeared and as expected these slow down the development. These issues were discussed at both the daily meeting and the weekly supervisor meeting depending on magnitude. Most issues was discussed on conceptual levels and if this wasn't enough help and knowledge was obtained from fellow students, teacher or researchers located at the school.\\
Among other these issues covers problems with the simulation tool multisim. While developing the power supply scheme throughout the system the we didn't understand why the hardware design didn't work as expected. We found that the program couldn't handle the different voltage references supplied to various components.

\section{Development approach}
The approach to development have changed from slowly dipping the feet into the water to rapid prototyping with almost new print layouts every week. This can be contributed to improved confidence in hardware design.