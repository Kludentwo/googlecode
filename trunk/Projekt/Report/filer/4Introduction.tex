\chapter{Introduction}

The goal of the project is to develop the technology to enable a product comprising of several sensors serially connected on a power line with overlaid communication, which is an extension of the original project description. This report will describe the phases, methods and solutions made and developed throughout the project.\\

The project is build upon the requirements and descriptions made in the preliminary project. Some of which has been changed or modified.\\

During the preliminary project a rough time schedule was composed, dividing the project into four phases.

\begin{itemize}
	\item System Design
	\item Technologies and research
	\item Development
	\item Documentation and finalizing
\end{itemize}

The overall purpose of the project is described in the course catalogue and is shortened to these points:

\begin{itemize}
	\item Use scientific research result and collect technical knowledge into a technical solution
	\item Develop new solutions
	\item Gather and evaluate new knowledge with regards to relevant engineering areas
	\item Complete routine engineering tasks.
	\item Communicate project results to professionals as well as customers
	\item Present project results orally using audio/visual communication tools
	\item Incorporate social, economic, environmental and health and safety consequences in a solution	
\end{itemize}

\section{Reading guide}
The report should be read as four sections, introduction, methods, results and conclusion.\\
The introduction covers the specification of the project by project formulation, system description, requirement specification and project scope.\\
Methods explains the approach to the project and solutions, planning and scheduling, work distribution and project structure.\\
The results covers the result of the project namely what have been made and what have been learned. Most subjects here will be described briefly but further information can be found in the more detailed descriptions in the documentation found on the CD.\\
The conclusion rounds up the project and weighs of expectations and goals.

\section{Glossary and abbreviations}
\begin{table}[H]
\centering
\begin{tabular}{|p{4cm}|p{7cm}|}
\hline
Term/abbreviation & Definition\\ \hline
CDU & Abbreviation of Central data unit, a key component in the system.\\ \hline
SN & Abbreviation of Sensor node, a key component in the system.\\ \hline
LPF & Abbreviation of Low Pass Filter. \\ \hline
HPF & Abbreviation of High Pass Filter. \\ \hline
PLL & Abbreviation of Phase Lock Loop. \\ \hline
IC & Abbreviation of Integrated Circuit. \\ \hline
PCB & Abbreviation of Printed Circuit Board. \\ \hline
DLM & Abbreviation of Data Length Multiplier. \\ \hline
CRC & Abbreviation of Cyclic Redundancy Check. \\ \hline
\end{tabular}
\end{table}