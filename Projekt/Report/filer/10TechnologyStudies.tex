\chapter{Technology studies}

The technology studies covers the studies leading up to the design and implementation. The technology studies help give an overview of how to realise the power line communication bus by providing good knowledge into the theory needed. During the technology studies the procedure were:
\begin{enumerate}
\item Discover literature
\item Create overview
\item Read and understand
\item Experiment
\end{enumerate}


\section{Protocol and communications}

\subsection{Communications}
The field of communication is very large, so to determine how to physically perform the communication in this system other methods were studied.\\
Commonly used for communcation is level shifting, this is known from many common communication protocols such as RS232, i$^2$C, SPI etc.\\
These physical protocol layers depend on synchronization either by including a clock or specifying a baudrate.\\



\subsection{Protocol}
To determine how the protocol should structured parallels to known communication protocols were made. The bus with multiple slaves/nodes have a lot of similarities physically with the i$^2$C protocol.\\
The i$^2$C protocol is a two wire system with multiple slaves, which responds to requests from the master. An i$^2$C transfer sequence is seen on the figure \ref{fig:i2cheader} below, it is found in the i$^2$C protocol specification \fixme{reference til i2c protocol specificationen på http://i2c.info/i2c-bus-specification}

\begin{figure}[H]
	\centering
	\includegraphics[width=.8\textwidth]{billeder/10technologystudies/7-bit-address-writing}
	\caption{i$^2$C protocol transfer sequence}
	\label{fig:i2cheader}
\end{figure}

The transfer sequence is initiated by a start sequence from the master, this alerts the slaves and they are now listening on the bus. The next step for the master is to transmit the address of the slave. Because several slaves are connected an addressing system must be utilized. Lastly the master tells whether it is a read or write command. \\
The same principles are applied to the power line communication bus where an addressing system and start sequence is used.\\

Another known protocol is the modbus protocol which is a widely used protocol. The modbus protocol has a start and addressing system, as the i$^2$C. But after the address a function code is transmitted. This code is used to tell the slave exactly what to do. Below on figure \ref{fig:modbusframe} is shown the modbus frame format.

\begin{figure}[H]
	\centering
	\includegraphics[width=.8\textwidth]{billeder/10technologystudies/modbusframe}
	\caption{Modbus frame/header}
	\label{fig:modbusframe}
\end{figure}

The function codes are standardized and covers a wide variety of uses, which covers all from simple request as, read discrete inputs, to diagnostic.

\subsection{The OSI model}
The structure of the communication is based upon the principle from the OSI model. The OSI model is an abbreviation of the Open Systems Intercommunication Model. The model helps describe and standardize the structure of communication systems. The OSI model has seven layers:
\begin{enumerate}
	\item Physical
	\item Data link
	\item Network
	\item Transport
	\item Session
	\item Presentation
	\item Application
\end{enumerate}

Each layer has a specific responsibility to enable the entire communication system. The layers utilized in the power line communication bus implemented in this project is, the physical layer, \fixme{reference til OSI modellen skal eventuelt bare drages i arkitekturen}


\section{Clock and data recovery}
Normally a data transmission consists of two elements, clock and data. In this system that is not possible and therefore the area of clock and data recovery was investigated.\\
Clock and data recovery is widely used in the field of data transmission. Especially in high bandwidth systems. Many of such systems use NRZ (Non-return to zero) line coding where in essence the clock is embedded in the data stream.

\subsection{What is clock recovery?}
Clock recovery is achieved essentially by extracting the clock from the incoming data stream. Due to the level shifts of the bitstream the clock is carried in the data.
\fixme{Skrives noget mere til hvad clock recovery er}

\subsubsection{How is clock recovery achieved?}


\subsection{The Phase-locked loop}
The phase-locked loop(PLL) comes in a large variety of designs but they all have one common conceptual diagram. A PLL consists of a phase-detector, a filter and a voltage controlled oscillator(VCO). Optionally it can contain a clock divider as well to speed up the VCO clock. Below on figure \ref{fig:conceptualpll} is shown the structure of the conceptual PLL.

\begin{figure}[H]
	\centering
	\includegraphics[width=.9\textwidth]{billeder/10technologystudies/conceptualpll}
	\caption{Conceptual PLL}
	\label{fig:conceptualpll}
\end{figure}

\subsubsection{Phase detector and loop filter}
The phase detector has two inputs. The input signal and the loop feedback signal. The phase detector compares the two input and produces a series of phase pulses depending on the phase difference of the two input signals. There is a vareity of different phase detector but two types are so common they are named type 1 and type 2 phase detectors. Each type of detector has an impact on how the loop locks and at what phase.
\\ \textbf{Type 1:}\\
The type 1 phase detector is basically a XOR gate. 

\begin{figure}[H]
	\centering
	\includegraphics[width=.6\textwidth]{billeder/10technologystudies/XORgate}
	\caption{XOR gate}
	\label{fig:XOR}
\end{figure}

The XOR gate outputs a digital high when the two input signals are different from one another and a digital low when the two input signals are in the same state. Below on figure \ref{fig:pd1_waveforms} is shown the waveforms produced by a type 1 phase detector.
\begin{figure}[H]
	\centering
	\includegraphics[width=.8\textwidth]{billeder/10technologystudies/PD1_waveforms}
	\caption{Phase detector type 1 waveforms}
	\label{fig:pd1_waveforms}
\end{figure}

The two input signals are out of frequency. And on the PD out line it is shown how this changes with regards to the two inputs.  The output of the low pass filter i used as an input to the VCO to either speed up the clock or slow it down.\\
When the PLL is in lock the LPF output will be $\frac{\text{V}_{\text{DD}}}{2}$. This effectively means that when the PLL locks on the In 1 signal the In 2 signal will be in exactly 90$^{\circ}$ deg phase.\\
\textbf{Type 2:}\\
The type 2 phase detector comprises of a phase comparator and a charge pump. The comparator is basically two D type flip-flops connected with and AND gate as seen on the figure \ref{fig:pd2_imp} below. The output from the phase comparator controls the charge pump to either charge or discharge the filter.

\begin{figure}[H]
	\centering
	\includegraphics[width=.7\textwidth]{billeder/10technologystudies/pd2_imp}
	\caption{Phase detector type 2 functional figure}
	\label{fig:pd2_imp}
\end{figure}

The advantage for the type 2 phase detector is the ability to detect a lead or a lag in frequency. Below on figure \ref{fig:pd2_waveform}, the waveform is illustrated for both leading and lagging. Depending on the amount of lead og lag the pump will be on/off for longer or lower periods of time.

\begin{figure}[H]
	\centering
	\includegraphics[width=1\textwidth]{billeder/10technologystudies/pd2_waveform}
	\caption{Phase detector type 2 waveforms}
	\label{fig:pd2_waveform}
\end{figure}

\subsubsection{Voltage controlled oscillator and Clock divider}
The voltage controlled oscillator serves to convert the voltage output of the low pass filter to an output frequency. Below on figure \ref{fig:VCO_func} is shown the conceptual relationship between voltage input and frequency output.

\begin{figure}[H]
	\centering
	\includegraphics[width=1\textwidth]{billeder/10technologystudies/VCO_functionality}
	\caption{Voltage controlled oscillator functionality}
	\label{fig:VCO_func}
\end{figure}



