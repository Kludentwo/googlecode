\chapter{Abstract}
The report for a bachelor project in electronic engineering at Aarhus School of Engineering, Aarhus University presents an explanation of the groups project: "Charcot-foot design and development of temperature measurement device". The purpose of the original project is to design and develop as system that can measure the temperature in different places on the foot of a patient suffering from charcot-foot. The modified version of the project entails designing a system with several sensors connected to a power line with overlaid communication. The focus is on developing the technology behind the power line communication bus.\\
A thorough documentation of the whole process, covering requirement specification, system architecture, design and implementation has been made. A series of internal tests are used as a baseline for improving the system.\\
The group has approached the project in an iterative manner in order to develop and improve the system. A SCRUM-like board has been used to structure the tasks and a time schedule was made to structure the different phases in the project.\\
A fully working proof of concept system has been made consisting of a central data unit and two sensor nodes. They communicate over the custom power line communication bus developed in this project. A chapter detailing the further development need is included in this report.\\
The project has netted both group members a better understanding of developing and hardware components as a whole.\\

\chapter{Resume}
