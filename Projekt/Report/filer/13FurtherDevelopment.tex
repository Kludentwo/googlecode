\chapter{Further development}
\textit{This chapter briefly suggests elements and thoughts of what work needs to be done to finalize the development of the product\fixme{jeg kan godt lide det som sætningen vil fortælle men der skal måske lige laves noget opdeling af sætningen og lidt ordstilling}. The order of sections below is random and does not suggest any prioritization.}
%This chapter seeks to explain the further development needed for this project to become a full product.\\

\section{Component selection}
The whole project was made with discrete components that were available from the schools electronics laboratory. This means that the prototype was developed with components that were not the most fitting or optimized for the project. To further improve the project some components will have to be changed for cheaper or more effective components or circuits. \\
An example could be improving the operational amplifier handling the incoming data amplifying with one with better slew rate, better signal-noise ratio and/or wider gain-bandwidth product might improve stability in the data transmission.

\section{Supplying the unit}
A full battery supply system will need to be designed. It must take the total power consumption over time in consideration in order to comply with the requirement specified in the original project formulation. The voltage to the bus must also be taken into consideration, see section \ref{sec:DO}.

\section{IC synthesizing}
A big part of the further development is the design and synthesis of an integrated circuit. The whole sensor node must be redesigned with the same blocks but different components and layouts. The finished sensor node IC will still need to have a small circuit board with a few capacitors to ensure stability.

\section{Application layer}
Another part is the application layer of the CDU. The functions and descriptions have been developed but the application will need to be written and optimised for this project to become an actual product. Further development on the CDU could be the ability to update the firmware through a connector placed somewhere in the project.

\section{Data acquisition}
The measurement system will also have to be improved. Either through a redesign or implementation of a know system with low power consumption and low noise. This could be a full project on its own.

\section{Design optimization}
\label{sec:DO}
Currently the voltage supply level on the sensor node are 3.3V. To lower power consumption and make it practically possible to have a lot of sensors serially connected it is needed to lower the operating voltage level throughout the entire sensor node. The group suggests a voltage maximum of about 2V for a sensor node. This enables for a lot more sensors then now or a lower voltage level to supply the bus. It also helps the battery power extension since the need for a high voltage can be inefficient.

Another design optimisation is the choice of µ-Controller to be used in the CDU. The current software only utilise about 4\% of the program memory and 7 (of a total of 85) of the general purpose pins. A smaller and cheaper µ-Controller will most certainly fit the project.