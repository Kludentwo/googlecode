\chapter{Project Formulation}

This chapter describes the project formulation forming the foundation for the project. The original project formulation was provided by DELTA and adopted by the group.

\section{Original project formulation}
This is only a short excerpt of the original project formulation, for the entire original project description see \fixme{Reference til bilag eller cd hvor den originale beskrivelse kan findes.}
\subsection{Background}
A small part of diabetic patients will develop Charcot-foot, a complication often introduced late in the illness. This complication immobilises patient and degrades quality of life.\\
It is believed that the charcot-foot is causing rise in the temperature of the foot/feet and that the heat signature is different from the/a healthy foot. Earlier projects have shown that temperature measurement can be used to evaluate on the condition of the foot in the rehabilitation phase.

\subsection{Project description}
The project should end in a prototype able to continually measure the temperature of the foot. The device should be worn by the patient throughout the daily life.\\
The project is expected to deliver a design of such a system, where focus should be on the temperature measurement device, although it is expected to describe the entire system.\\
Expected results:
\begin{itemize}
	\item Prototype
	\item Requirement specification of prototype
	\item System description including interface descriptions
\end{itemize}


\section{Modified project formulation}
This project description is an extension of the original project description. 
\subsection{Background}
The original project formulation was not fulfilling to the group with regards to new knowledge and technology. Therefore the project was modified to accommodate the groups needs.

\subsection{Project description}
The project should seek a solution incorporating smart sensors.

