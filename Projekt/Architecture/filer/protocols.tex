\chapter{Protocols}
This chapter serves to describe the communication protocol. Each sensor has an unique address. The sensors responds to function codes shown in table ~\ref{table:functioncodes}.

\section{Function codes}
\begin{table}[H]
\centering
\begin{tabular}{|l|l|l|}
	\hline
	Function code & Name & Data \\ 
	\hline
	0001 	& GetInfo & Error codes and Type \\
	\hline
	0010	& GetData & Error codes and 12 bit of data from sensor \\
	\hline
\end{tabular}
\caption{Function code table with names and data explaination}
\label{table:functioncodes}
\end{table}

\section{Communication sequence}
The communication consist of a write sequence followed by a read sequence.\\
Writing to a sensor node takes a message of variable length. The message is comprised of a start sequence, an address, a function code, a data length multiplier (DLM), data and lastly a CRC code. "CRC8 bit, normal" is used for CRC. This is shown in table ~\ref{table:stdmsgtosensor}. The message length tells how many bytes of data will be send. This way the receiver knows how much data it must receive. Once the CDU has written to the sensor node, the sensor node responds using the the same format. \\

\begin{table}[H]
\centering
\begin{tabular}{|l|l|l|l|l|l|}
	\hline
	Start Sequence & Address & Function Code & DLM & Data & CRC  \\ \hline
	1 nibble & 1 nibble	& 1 nibble & 1 nibble & n bytes & 1 byte\\
	\hline
\end{tabular}
\caption{Message format for writing and reading}
\label{table:stdmsgtosensor}
\end{table}
The start sequence is always "0110".\\

The data coincides with the function code. This is further explained below.
\subsection{GETINFO function code}
Data is read as in table ~\ref{table:Datagetinfo}.
\begin{table}[H]
\centering
\begin{tabular}{|l|l|}
	\hline
	Errors & Type \\ \hline
	1 nibble & 1 nibble \\
	\hline
\end{tabular}
\caption{Data format for function code GETINFO }
\label{table:Datagetinfo}
\end{table}
\subsection{GETDATA function code}
Data is read as in table ~\ref{table:Datagetdata}.
\begin{table}[H]
\centering
\begin{tabular}{|l|l|}
	\hline
	Errors & Data \\ \hline
	1 nibble &  12 bit \\
	\hline
\end{tabular}
\caption{Data format for function code GETDATA }
\label{table:Datagetdata}
\end{table}
An example of 12 bit data from a sensor of the type temperature sensor is "432" or "000110110000" which corresponds to 43.2 degrees Celsius.

\subsection{Type}
\begin{table}[H]
\centering
\begin{tabular}{|l|l|l|}
	\hline
	Type & Bitpattern & Description \\ 
	\hline
	No type 	& 0000 & Invalid sensor type \\
	\hline
	Temperature sensor & 0001 & Temperature sensor node \\
	\hline
	Free Space for future sensors & 0010-1111 & Not in use \\ \hline
	
\end{tabular}
\caption{Type table with bitpattern and description}
\label{table:typetable}
\end{table}

\subsection{Errors}
\begin{table}[H]
\centering
\begin{tabular}{|l|l|l|}
	\hline
	Status & Bitpattern & Description \\ 
	\hline
	Communcation error 	& 0001  & The correct address was received but no function code. \\
	\hline
	Unknown function code  & 0010  & Function code is not known by sensor. \\
	\hline
\end{tabular}
\caption{Error table with bitpattern and description}
\label{table:errortable}
\end{table}




