\chapter{Structural view}

\section{Introduction}
This chapter describes the structural architecture of the system in a top down approach.\\
The blocks are derived from previous documentation.


\section{System overview}
The system consist of a central data unit(CDU) and n number of sensor nodes (SN).\\
Power and communication to all sensors are carrie through the custom powerline communication bus on which all sensor nodes are serially connected. Below, on figure \ref{fig:systembdd}, is shown a figure of the system overall blocks and interfaces.
\begin{figure}[hbpt]
	\centering
	\includegraphics[width=.9\textwidth]{billeder/systembdd}
	\caption{System overview}
	\label{fig:systembdd}
\end{figure}

\subsection{General interfaces}

\begin{table}[H]
	\begin{center}
		\begin{tabular}{|p{2.5cm}|p{1cm}| p{9cm}| }
			\hline
			\textbf{Block} & \textbf{IO} & \textbf{Description} \\\hline
			\multirow{2}{2.5cm}[-2em]{CDU} & B+& The positive voltage out from the CDU is supplied to the first sensor. The level must be sufficient to supply the number of sensors connected.\\ \cline{2-3}
 & B-  & The negative port on the CDU is connected to the last sensors in the system and thereby ends the sensor loop. \\ \hline
			\multirow{2}{2.5cm}[-2em]{Sensor node} & S+ &The In\_pos is where the high potential voltage is supplied to the sensor node. \\ \cline{2-3}
& S- & The negative port at the sensor node is looped to the possitive port on the next sensor or on the "B-" port at the CDU. \\ \hline
		\end{tabular}
	\end{center}
\end{table}

\subsection{Block responsibility}
\subsubsection{CDU}
The CDU contacts each sensor to collect the data values. This is then stored for later extraction.

\subsubsection{Sensor node n}
The sensor node measures temperature.

\subsection{System layers}
The system is divided into four layers, as shown on figure \ref{fig:system_layers}

\begin{figure}[H]
	\centering
	\includegraphics[width=.7\textwidth]{billeder/system_layers}
	\caption{System layers}
	\label{fig:system_layers}
\end{figure}

This architecture simplifies the communication process and clearly separates responsibility. This structure is based upon the idea of the OSI model [1].

\subsubsection{Application layer}
The application layer is only application specific functionality. This covers all functionality enabling the device to full fill the use cases, but does not handle any communication.

\subsubsection{Protocol layer}
The protocol layer handles the protocol specific functionality. This covers composing messages which complies with the protocol as well as interpreting incoming messages.

\subsubsection{Line coding layer}
The line coding layer converts the messages to the right sequence to be transmitted on the physical layer. This also covers converting incoming messages into buffers to be interpreted by the protocol layer.

\subsubsection{Physical layer}
The physical layer converts the digital message to an analog signal carried on the power line and vice versa.

\section{Detailed block overview}
Below are detailed figures for each block in the system.\\
The blocks in the figures are conceptual blocks and most of them consist of both software and hardware.\\
All the blocks are needed to fulfil the current functionality and requirements of the system. Supporting blocks might be added if the functionality does not fit into the current blocks.

\subsection{CDU}

\begin{figure}[hbpt]
\centering
\includegraphics[width=.8\textwidth]{billeder/CDU_IBD}
\caption{Internal Block Diagram of the CDU}
\label{CDU_IBD}
\end{figure}

\subsubsection{Block description}

\textbf{Power supply:}\\
The power supply feeds the necessary power to the modules on which it is connected.\\

\textbf{Sensor power supply:}\\
The Sensor power supply feeds the power needed to operate all sensors connected to the system.\\

\textbf{µ-Controller:}\\
The µ-Controller runs the main application. Main objectives:\\
$\bullet$ Keep track of sensors\\
$\bullet$ Collect data\\
$\bullet$ Store data\\
$\bullet$ Manage communication protocol\\

\textbf{Sensor communication:}\\
The Sensor communication interfaces with the sensor power supply to modulate the power line to the sensor nodes.\\
The sensor communication block has no intelligence and the protocol is handled by the µ-Controller block.\\

\textbf{Memory:}\\
The memory stores all data collected by the µ-Controller. The memory can be detachable or some kind of eeprom/flash.\\

\textbf{PC communication:}\\
The PC communication is an interface to a PC.\\

\subsubsection{Signal description}
\begin{table}[H]
	\centering
    \begin{tabular}{|l|p{9cm}|}
    \hline
    Sensor power & This signal contains the power to the Sensor Power supply block. \\
    \hline
    Power \& communication & The signal contains the power for the sensor nodes along with the communication signal. \\
    \hline
    µC power & This signal contains the power for the µ-Controller block. \\
    \hline
    Com power & This signal contains the power for the Sensor communication block. \\
    \hline
    A Sensor com & This signal contains the analog levels for sensor communication \\
    \hline
    D Sensor com & This signal contains the digital levels for sensor communication \\
    \hline
    Pc data & This signal contains the µ-Controller levels of Pc data \\
    \hline
    Mem data & This signal contains data to and from memory \\
    \hline
    Pc data* & This signal contains the PC levels of Pc data \\
    \hline
    \end{tabular}
    \caption{Signal description for CDU block}
    \label{CDUsignaldes}
\end{table}

\subsection{Sensor node n}

\begin{figure}[hbpt]
\centering
\includegraphics[width=.8\textwidth]{billeder/Sensor_IBD}
\caption{Internal Block Diagram of the sensor}
\label{Sensor_IBD}
\end{figure}

\subsubsection{Block description}

\textbf{Power supply:}\\
The power supply located in the sensor regulates the voltage levels to fit the need of the individual blocks in the sensor.\\

\textbf{Communication:}\\
The communication block demodulates the communication on the input line and feeds it to the sensor logic. It also handles the conversion from a digital stream from the logic handler to a physical response to the CDU.\\

\textbf{Logic handler;}\\
The logic handler interprets the incomming communication and responds accordingly.\\
It also handles communication with the data acquisition block as well as converting the data to a temperature value.\\

\textbf{Data acquisition:}\\
The data acquisition block is a complete measurement system with a digital interface to the logic handler. It serves to measure the physical biometric parameter.\\

\subsubsection{Signal description}
\begin{table}[H]
	\centering
    \begin{tabular}{|l|p{9cm}|}
    \hline
    Power \& CDU com & This signal contains the power, clock and data from the bus. \\
    \hline
   	Communication supply & This signal contains the supply power for the communication block. \\
    \hline
    CDU com & This signal contains the communication to and from the CDU. \\
    \hline
    Logic supply & This signal contains the supply power for the logic handler block. \\
    \hline
    Data acquisition com & This signal contains the communication to and from the data acquisition block. \\
    \hline
    Data acquisition supply & This signal contains the supply power for the data acquisition block. \\
    \hline
    \end{tabular}
    \caption{Signal description for SN block}
    \label{SNsignaldes}
\end{table}

